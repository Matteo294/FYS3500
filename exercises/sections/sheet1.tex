\subsection*{Exercise 1}

\subsubsection*{a)}
\begin{gather*}
    Z(Th) = 90 \quad\Rightarrow\quad N(^{232}Th) = 142 \\
    Z(U) = 92 \quad\Rightarrow\quad N(^{235}Th) = 143
\end{gather*}

\subsubsection*{b)}
$$ ^{20}Ne $$

\subsubsection*{c)}
$$ ^{82}Kr $$

\subsubsection*{d)}
Noone

\subsubsection*{e)}
Decay mode

\subsection*{Exercise 2}
Suppose that an elementary particle \emph{a} is a bound state of two other particles \emph{b} and \emph{c}
respectively with parities $\pi_b$ and $\pi_c$. Then we can assign a parity to the particle \emph{a} as
\begin{equation}
    \pi_a = \pi_b \pi_c \, (-1)^{l_a}
\end{equation}
where $l$ the denotes the total angular momentum number of the particle \emph{a}.
In general, if we have a decay process of the type $$a + b \rightarrow c + d$$ it can be proved that parity
is conserved, so that the following equality holds 
\begin{equation}
     \pi_a \pi_b \, (-1)^{l_{a,b}} = \pi_c \pi_d \, (-1)^{l_{b,c}}
\end{equation}
where $l_{a,b}$ and $l_{c,d}$ denotes respectively the total angular momentum number of the couple $a, b$ 
and $c, d$.

\subsubsection*{a)}
Since the fermions have conventionally intrinsic parity $+1$, the corresponding antiparticles
should have parity $-1$. The pion spin is $0$, hence
$$\pi_{\pi} = \pi_u \pi_{\bar d} \, (-1)^0 = -1$$

\subsubsection*{b)}
The product of the two intrinsic parities is $+1$, while the contribute of the total angular momentum
is $-1$, hence $$\pi_f = -1$$

\subsection*{Exercise 3}

\subsubsection*{a)}
In the mass radius we have to keep track of two effects: the repulsion between the charged protons and the
proton-neutron attraction. The first effect makes the protons to spread widely, while the second provides
a strong attraction causing the neutrons to stick to the protons, hence explaining the equality in the two 
radii (in first approximation).

\subsubsection*{b)}
The nuclear force is strong for very short ranges (r < 1f) but decays rapidly as the distance increases.
As a consequence, for heavy and large atoms, the nuclear attraction force is just enough to counteract
the proton-proton repulsion. Accoroding to quantum mechanics, each $alpha$ particle can be seen as a
particle subject to a wall potential, and the probability to be found outside the wall is non-zero
and in particular depends on the atomic radius. Hence, an analysis of such probability provides
an estimation for radius value. \\
The reason why this phenomenom is more likely to occur for $\alpha$ particles rather than protons
is explained by the binding energy, presented later in the notes