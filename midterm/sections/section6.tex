I use the definition of decay stated \href{https://en.wikipedia.org/wiki/Particle_decay}{here}
\subsection*{a}
\begin{table}[htbp]
    \centering
    \begin{tabular}{cccccc}
        \toprule
            Process & Type & Interaction & Allowed & Suppressed & Reason not allowed \\
        \midrule 
            $1$ & decay & electroweak & yes & no & \\
            $2$ & & & no & & charge \\
            $3$ & decay & electroweak & yes & yes & \\
            $4$ & decay & weak & yes & yes & \\
            $5$ & decay & strong & yes & yes & \\
            $6$ & reaction & electroweak + strong & yes & yes & \\
            $7$ & decay & weak & yes & yes & \\
            $8$ & & & no & & lepton number \\
        \bottomrule
    \end{tabular}
\end{table}
\subsection*{b}
\subsection*{c}
\begin{itemize}
    \item Let us consider the decay 1. One can estimate the average lifetime $\tau$ of $\psi(3086)$ as the squared inverse of the coupling constant, hence
    \begin{equation*}
        \tau \approx 
    \end{equation*}
    The experimental known lifetime is \footnote{}. \\
    \item For the decay $3$ the lifetime can be estimated as $\tau \approx \frac{1}{G_F^2} \approx 10^{-12}$ which is compatible with the \href{http://hyperphysics.phy-astr.gsu.edu/hbase/Particles/qrkdec.html}{experimental value}
    \item For the decay $7$ the lifetime can be estimated as 
\end{itemize}
\subsection*{d}
\subsection*{e}