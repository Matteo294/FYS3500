\subsection*{a}
This result can be explained by means of the nuclear shell model. According to this model each sub-level of the nucleus can be identified by 4 quantum numbers $\{n, j, l, m_l\}$. For fixed $\{n, j, l\}$ 
one has that the $2l+1$ sub-levels given by $m_l = -l, -l+1, \dots, l-1, l$ are degenerate and each of this sub-levels can contain two nucleons with different spin z-projrection ($m_s = \pm 1$). \\
At first one can notice that if one sub-shell is filled with two nucleons (of the same type) then the spin contibution of that sub-shell is 0. Having an even number of protons and neutrons means that 
they combine in couples in the sub-shell in a way such that the contribution to the spin of each sub-shell is 0. \\
For what concerns the parity, each particle in the nucleus gives a contribute of $(-1)^l$ (the intrinsic parity of the nucleons is $+1$) and the total parity is the product of single parities. 
The product of the parities of two protons (neutrons) in the same sub-shell is always $+1$ since they, in particular, share the same quantum number $l$. Hence it is legitimate to expect an even-even nucleus
to be in the state $0^+$.
\centerline{\textbf{Answer}: spin $0$, parity $+$}.