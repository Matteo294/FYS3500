\subsection*{a}
The mass of the atom is given by
\begin{equation*}
    M(^{48}\text{CA}) = Zm_P + (A-Z)m_N - \frac{B(A, Z)}{c^2}
    \label{eq:semi_empirical_mass_formula}
\end{equation*}
where $B$ indicates the binding energy, which in turns can be calculated via the semi-empirical formula (atomic units)
\begin{equation}
    B(A, Z)= a_v A + a_s A^{2/3} + a_c \frac{Z(Z-1)}{A^{1/3}} - a_{sym} \frac{(A - 2Z)^2}{A} + \delta (A, Z)
    \label{eq:semi_empirical_BE_formula}
\end{equation}
where 
\begin{equation*}
    \delta(A, Z) = 
    \begin{cases}
        \frac{a_P}{A^{3/4}} \qquad \text{if Z and N are even} \\
        0 \qquad \text{if A is odd} \\
        -\frac{a_P}{A^{3/4}} \qquad \text{if Z and N are odd} 
    \end{cases}
\end{equation*}
One can simply pop in the values of $A$ and $Z$ into the formula using the fitted coefficients (source Wikipedia) in units of $MeV/c^2$
\begin{equation*}
    a_v = 15.8 \qquad a_s = 18.3 \qquad a_c = 0.714 \qquad a_{sym} = 23.2 \qquad a_P = 12
\end{equation*}
and obtains $B(A, Z) = 621.71~MeV/c^2$. Hence, inserting the result into \ref{eq:semi_empirical_mass_formula} one obtains
\begin{equation*}
    M(^{48}\text{Ca}) \simeq 44451.55~MeV/c^2 \simeq 47.3~u
\end{equation*}
Even though the result is closet to the experimental value there is a small difference. One can justify this difference by observing that the SEMF is 
not a complete theoretical model, but rather a fit to experimental data of a model with some theoretical fundations. As such, it is reasonable that the model
can give an overall good description of the data but cannot perfectly describe each of them: this is especially the case of lighter atoms for which
the quantum shell structures of the nucleus should be taken into account.

\subsection*{b}
Since $B(A, Z) = \sum_i B_i(A_i, Z_i)$ one can calculate the contribution of the binding energy of the last neutron as 
\begin{equation*}
    B_{neutron} = B(A, Z) - B(A-1, Z)
\end{equation*}
And using expression \ref{eq:semi_empirical_BE_formula} to calculate the two quantities one ends up with an estimation for the energy required to extract the neutron
\begin{equation*}
    B_{neutron} \approx 15.69~MeV/c^2
\end{equation*}

\subsection*{c}
The most stable atom corresponds to the maximum point of the binding energy as a function of $Z$. One can find such point by imposing $\frac{dB(Z)}{dZ} = 0$. By the way 
one problem arises, that is that the last therm of the binding energy depends on the value of $Z$ and we do not know it in advance. Hence the approach I chose was to set that term to 0 and
find the maximum point $Z$ : this gives a non-integer term and the correct $Z$ is the one closer to this integer. Hence
\begin{equation*}
    0 = \frac{dB(Z)}{dZ} = \frac{a_c}{A^{1/3}} (2Z - 1) + \frac{4a_{symm}}{A} (A - 2Z)
\end{equation*}
leads to 
\begin{equation*}
    Z = \frac{a_c A^{2/3} - 4a_{symm}A}{2a_c A^{2/3} - 8a_{symm}} \simeq 
\end{equation*}
