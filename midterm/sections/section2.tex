\subsection*{a}
The mass of the atom is given by
\begin{equation*}
    M(^{48}\text{CA}) = Zm_P + (A-Z)m_N - \frac{B(A, Z)}{c^2}
    \label{eq:semi_empirical_mass_formula}
\end{equation*}
where $B$ indicates the binding energy, which in turns can be calculated via the semi-empirical formula (atomic units)
\begin{equation}
    B(A, Z)= a_v A + a_s A^{2/3} + a_c \frac{Z(Z-1)}{A^{1/3}} - a_{sym} \frac{(A - 2Z)^2}{A} + \delta (A, Z)
    \label{eq:semi_empirical_BE_formula}
\end{equation}
One can simply pop the values of $A$ and $Z$ into the formula using the coefficients
\begin{equation*}
    a_v = \qquad a_s = \qquad a_c = \qquad a_{sym} = \qquad a_p = \qquad
\end{equation*}
and obtains $B(A, Z) = val$. Hence, inserting the result into \ref{eq:semi_empirical_mass_formula} one obtains
\begin{equation*}
    M(^{48}\text{CA}) \simeq val
\end{equation*}

\subsection*{b}
Since $B(A, Z) = \sum_i B_i(A_i, Z_i)$ one can calculate the contribution of the binding energy of the last neutron as 
\begin{equation*}
    B_{neutron} = B(A, Z) - B(A-1, Z)
\end{equation*}
And using expression \ref{eq:semi_empirical_BE_formula} to calculate the two quantities one ends up with an estimation for the energy required to extract the neutron
\begin{equation*}
    B_{neutron} = B(A, Z) - B(A-1, Z) \approx 
\end{equation*}