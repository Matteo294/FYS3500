\subsection*{a}
In the resting frame of the Higgs boson the invariant mass is (atomic units)
\begin{equation}
    W = \sqrt{(\sum_i E_i)^2 - |\sum \ve{p}_i|^2} = E_H = m_H
    \label{eq:W_Higgs_frame}
\end{equation}
The invariant mass is a Lorent invariant, hence this quantity is the same that one observer would measure in the laboratory frame. 
More specifically, one can calculate $W$ after the process using the electrons data
\begin{equation*}
    W = \sqrt{(\sum_i E_i)^2 - |\sum \ve{p}_i|^2} = \sqrt{(\sum_i E_i)^2} = \sum \sqrt{m_i^2 + p_i^2}
\end{equation*}
By equating this expression to \ref{eq:W_Higgs_frame} and inserting the given values together with the electrons mass one obtains
\begin{equation*}
    m_h \simeq 126.2~GeV/c^2
\end{equation*}

\subsubsection*{b}
In principle each pair of electron-positron can be generated by the $Z$ boson, in the sense that no rules would be violated. 
In other words there are $4$ different ways to combine the couples of electron-positron to the bosons
\begin{enumerate}
    \item $Z \rightarrow e^+_1 + e^-_1, \quad Z^* \rightarrow e^+_2 + e^-_2$
    \item $Z \rightarrow e^+_1 + e^-_2, \quad Z^* \rightarrow e^+_2 + e^-_1$ 
    \item $Z \rightarrow e^+_2 + e^-_1, \quad Z^* \rightarrow e^+_1 + e^-_2$
    \item $Z \rightarrow e^+_2 + e^-_2, \quad Z^* \rightarrow e^+_1 + e^-_1$
\end{enumerate}
One can now impose the conservation of the $4-$momentum in the bosons' decay and in particular, for the $Z$ boson, this means 
\begin{gather*}
    E_Z = \sqrt{m_Z^2 + p_Z^2} = E_{e^+} + E_{e^-} = \sqrt{m_{e^+}^2 + p_{e^+}^2} + \sqrt{m_{e^-}^2 + p_{e^-}^2} \\
\end{gather*}
and by rearrangin terms
\begin{gather*}
    p_Z^2 = (E_{e^+} + E_{e^-})^2 - m_Z^2
\end{gather*}
this equation gives the value of the momentum that the $Z$ boson must have in order to satisfy the conservation of the energy.
By explicit calculation for the $4$ cases one obtains 
\begin{enumerate}
    \item $p_Z^2 \simeq -2414~GeV/c$
    \item $p_Z^2 \simeq -7350~GeV/c$
    \item $p_Z^2 \simeq 744~GeV/c$
    \item $p_Z^2 \simeq  -5872~GeV/c$
\end{enumerate}
hence only in the third case the particle can be regarded as physical and not virtual.

\subsection*{c}
\subsection*{d}