\subsection*{a}
In the resting frame of the Higgs boson the invariant mass is (atomic units)
\begin{equation}
    W = \sqrt{(\sum_i E_i)^2 - |\sum \ve{p}_i|^2} = E_H = m_H
    \label{eq:W_Higgs_frame}
\end{equation}
The invariant mass is a Lorent invariant, hence this quantity is the same that one observer would measure in the laboratory frame. 
More specifically, one can calculate $W$ after the process using the electrons data (laboratory frame)
\begin{equation*}
    W = \sqrt{(\sum_i E_i)^2 - |\sum \ve{p}_i|^2}
\end{equation*}
By equating this expression to \ref{eq:W_Higgs_frame} and inserting the given values together with the electrons mass one obtains
\begin{equation*}
    m_h \simeq 125.20~GeV/c^2
\end{equation*}

\subsubsection*{b}
In principle each pair of electron-positron can be generated by the $Z$ boson, in the sense that no rules would be violated. 
In other words there are $4$ different ways to combine the couples of electron-positron to the bosons
\begin{enumerate}
    \item $Z \rightarrow e^+_1 + e^-_1, \quad Z^* \rightarrow e^+_2 + e^-_2$
    \item $Z \rightarrow e^+_1 + e^-_2, \quad Z^* \rightarrow e^+_2 + e^-_1$ 
    \item $Z \rightarrow e^+_2 + e^-_1, \quad Z^* \rightarrow e^+_1 + e^-_2$
    \item $Z \rightarrow e^+_2 + e^-_2, \quad Z^* \rightarrow e^+_1 + e^-_1$
\end{enumerate}
One can now impose the conservation of the $4-$momentum in the bosons' decay and in particular, for the $Z$ boson, this means 
\begin{gather*}
    E_Z = \sqrt{m_Z^2 + p_Z^2} = E_{e^+} + E_{e^-} = \sqrt{m_{e^+}^2 + p_{e^+}^2} + \sqrt{m_{e^-}^2 + p_{e^-}^2} \\
\end{gather*}
and by rearrangin terms
\begin{gather*}
    p_Z^2 = (E_{e^+} + E_{e^-})^2 - m_Z^2
\end{gather*}
Every physical particle satisfies this relation, and virtual particles do not (that is why they are called "off-shell"). Hence one can assume the true mass of $Z$ and check if the corresponding 
momentum satisfies this relation (or it is "on-shell") or, equivalently, can assume the momentum by the conservation law and calculate the mass through the last relation 
which should results greater or equal than the true mass. I chose the first approach.
By explicit calculation for the $4$ cases, inserting the experimental value of the \href{https://en.wikipedia.org/wiki/W_and_Z_bosons}{Z boson mass}, one obtains 
\begin{enumerate}
    \item $p_Z^2 \simeq -2414~GeV/c$
    \item $p_Z^2 \simeq -7350~GeV/c$
    \item $p_Z^2 \simeq 744~GeV/c$
    \item $p_Z^2 \simeq  -5872~GeV/c$
\end{enumerate}
hence only in the third case (that is the one presented in the text of the exercise) the particle can be regarded as physical and not virtual.

\subsection*{c}
The three equations needed to solve the problem are the invariance of the invariant mass $W$ and the shell relation $E^2 = p^2 + m^2$. 
More specifically
\begin{equation*}\begin{cases}
    m_H = E_Z + E_{Z^*} \\
    E_Z^2 = m_Z^2 + p_Z^2 \\
    E_{Z^*}^2 = m_{Z^*}^2 + p_Z^2
\end{cases}\end{equation*}
this system can be solved for $E_Z, E_{Z^*}, p_Z$ and the two energies are
\begin{gather*}
    E_{Z^*} = \frac{m_H^2 + m_{Z^*}^2 - m_Z}{2m_H} \\
    E_Z = m_H - \frac{m_H^2 + m_{Z^*}^2 - m_Z}{2m_H}
\end{gather*}

\subsection*{d}
Every quantity can in principle be computed by the conservation of $4-$momentum. This means that 
all the four components must be conserved in each step of the process and in the center of mass frame this means that
\begin{equation*}
    (m_H, \ve 0) = (E_Z + E_{Z^*} \, , \, \ve p_Z + \ve p_{Z^*}) = \left(\sum_i E_i \, , \sum_i \ve{p}_i \right)
\end{equation*}
where the last summation index runs over the electrons. In addition, for each particle, one can use the shell relation $E = \sqrt{p^2 + m^2}$.
All this equations can be put together and the system of equation can be solved for the energies of the electrons.