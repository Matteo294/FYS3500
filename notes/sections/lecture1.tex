\subsection*{Parity}
The action of the parity operator $\mathcal{P}$ is defined on the coordinate vector as $$ \mathcal{P} \ket{x} = \ket{-x}$$.
For a generical state $\ket{\psi}$ it automatically follows that 
\begin{equation*}
    \mathcal{P}\ket{\psi} = \mathcal{P} \, \int dx \, \ket{x} \bracket{x}{\psi} = \int dx \, 
    \ket{-x} \bracket{x}{\psi} = \int dx' \, \ket{x'} \bracket{-x'}{\psi}
\end{equation*}
so that $$\bracketX{x}{\mathcal{P}}{\psi} = \psi(-x)$$
In spherical coordinates the action of the parity operator causes 
$r \rightarrow r$, $\theta \rightarrow \pi - \theta$ and $\phi \rightarrow \pi + \phi$. \\
Suppose now that $\ket{\psi}$ is an eigenstate of the parity operator so that 
$$\mathcal{P} \ket{\psi} = c \ket{\psi}$$
By applying $\mathcal{P}$ to both member of the last expression, one obtains that
\begin{equation}
     \mathcal{P}^2 \ket{\psi} = c^2 \ket{\psi}
     \label{eq:paritysquared1}
\end{equation}
but also, from what previously said
\begin{equation}
    \mathcal{P}^2 \ket{\psi} = \mathcal{P} \int dx \, 
    \ket{-x} \bracket{x}{\psi} = \int dx \, 
    \ket{-x} \bracket{x}{\psi} = \ket{\psi}
    \label{eq:paritysquared2}
\end{equation}
and combining \ref{eq:paritysquared1} and \ref{eq:paritysquared2} we conclude that $c = \pm 1$.
If the eigenvalue is $+1$ ($-1$) we say that the corresponding eigenstate has even (odd) parity.

\subsection*{Cross section}
Let us define the flux $J$ as a quantity that quantifies the rate of production of new particles
in the experiment (in the sense of things coming out of the target)
\begin{equation*}
    J = n_b \, v_i
\end{equation*}
where $n_b$ is the number density of particles in the beam (number of particles per unit volume) and 
$v_i$ is the speed of each particle in the rest frame of the target.
$J$ identifies the number of particles per second per unit area in the beam
\begin{equation*}
    \left[J\right] = m^{-2}s^{-1}
\end{equation*}
Call $N$ the number of particles in the target "illuminated" by the beam: we can then guess that
the number of scattered particle per second per unit area is of the form
\begin{equation*}
    W_r = J N \sigma_r
\end{equation*}
where $\sigma_r$ is a proportionality constant called \emph{cross section}. One has that 
\begin{equation*}
    \left[\sigma_r\right] = m^2
\end{equation*}
The cross section can be thought as the equivalent area that allows the process to occur. 
For example if both the target and the beam consist in a single particle, the equivalent area
at which the collison may occur is a circle of radius $2r$, hence $\sigma_r = \pi\left(2r\right)^2 
= 4\pi r$. To visualise it keep one sphere fixed, put the other
in contact and move it around keeping one contact point: this is the area that allows the collison.

\subsection*{Lifetimes}
Decay is a stochastic process: hence we can only know the probability of an atom to decay.
By observing a population of nuclei one describe the behaviour of the systems in therms of a
differential equation
\begin{equation*}
    \text{Activity} \ \equiv \ -\frac{dN}{dt} = \lambda N
\end{equation*}
from which
\begin{equation*}
    N(t) = N_0 e^{-\lambda t}
\end{equation*}
the time at which the population is reduced by a factor $2$ is 
$$t_{N/2} = \frac{1}{\lambda} \ln 2 \equiv \tau \ln 2$$
and we call $\tau$ (or equivalently $1/\lambda$) the decay characteristic time.

\subsection*{Decay width, branching ratio}
The probability distribution over time of the decay to occur is normally a gaussian-like distribution.
Let us then define a parameter $\Gamma$ called the \emph{decay width} that is the time difference
of the two points in which we have $N(t) = N/2$ (otherwise called FWHM). \\
If a particle may decay into multiple particles, we define a probability distribution for each 
event, and we can calculate the factor $\Gamma$ for each of them (call it $\Gamma_k$). We then define the
\emph{branching ratio} as $$B_k = \frac{\Gamma_k}{\Gamma}$$ to compare the different cases.

\subsection*{Q-value}
The Q-value express the mass energy difference between the states before and after the decay.
For example suppose that a particle $x$ may decay into $x$ and $z$. Then the Q-value for this event
is $$Q = \left(m_x - m_y - m_z\right) c^2$$.
If Q is positive energy is release during the process, otherwise we should add energy in order to 
make the decay happen.